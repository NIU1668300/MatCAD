\documentclass[11pt]{article}
\RequirePackage{etex}
%\pagestyle{empty}
\usepackage[activeacute,spanish,american]{babel}
\usepackage[utf8]{inputenc}%Para usar los acentos normalmente.
\usepackage[T1]{fontenc}
% \usepackage[usenames,dvipsnames,svgnames]{xcolor}
\usepackage{fullpage}
\usepackage{graphicx}
\usepackage{eurosym}
\usepackage{url}
\urlstyle{same}

\usepackage[makestderr]{pythontex}
% \restartpythontexsession{\thesection}

%%%%%%%%%%%%%%%%%%%%%%%%%%%%%%%%%%%%%%%%%%%%%%%%%%%%%%%5

\usepackage{tkz-base}
\usepackage[framemethod=TikZ]{mdframed}
\usepackage[most]{tcolorbox}
\usepackage{helvet, amssymb,amsmath,latexsym}
\usepackage{amsthm}
\usepackage{mathtools}
\usepackage{textcomp}
\usepackage{multienum}
\usepackage[inline,shortlabels]{enumitem}
\usepackage{multicol}
% \usepackage{gensymb}
\providecommand{\norm}[1]{\left\lVert #1 \right \rVert}
\providecommand{\abs}[1]{\left\lvert #1\right\rvert}
\usepackage{color,soul}%permite texto y subrayar en color.
\input{/home/samuel/Documents/Latex/Colores.tex}
% \usepackage[pdftex]{graphicx}
%Dimensiones
\usepackage[a4paper,left=2cm,right=1.5cm, top=1.5cm, bottom=3cm,includehead]{geometry}
%%%%%%%%%%%%%%%%%%%%%%%%%%%%%%%%%%%%%%%%%%%%%%%%%%%%%%%%%%%%%%%%%%%%%%%%%%%%%%%%%%%%%%%%%%%%
%Nombres de conjuntos y comandos propios.
\providecommand{\norm}[1]{\left\lVert #1 \right \rVert}
\providecommand{\abs}[1]{\left\lvert #1\right\rvert}
\newcommand{\C}{\mathbb{C}}
\newcommand{\N}{\mathbb{N}}
\newcommand{\Q}{\mathbb{Q}}
\newcommand{\Z}{\mathbb{Z}}
\newcommand{\R}{\mathbb{R}}
\newcommand{\I}{\mathbb{I}}
\newcommand{\prob}{\mathbb{P}}
\newcommand{\omg}{\omega}
\newcommand{\OMG}{\varOmega}
% \usepackage{esvect}
% \renewcommand{\vec}{\vv}
 \usepackage{pgf,tikz}
\usetikzlibrary{arrows.meta,arrows}
\usetikzlibrary{shadows}
\usetikzlibrary{shapes}
\usetikzlibrary{decorations.pathmorphing}
\usetikzlibrary{shapes.multipart}
\usetikzlibrary{chains}
\usetikzlibrary{scopes}
\usetikzlibrary{matrix}
\usetikzlibrary{positioning,automata,calc}
 %\usepackage{framed}
 %\usepackage[framed, amsthm,thmmarks,thref]{ntheorem}
 %\usepackage{tkz-tab,tkz-euclide,tkz-fct,tkz-linknodes}
 \usepackage{tkz-tab,tkz-euclide}
% \usetkzobj{all}
%%%%%%%%%%%%%%%%%%%%%%%%%%%%%%%%%%%%%%%%%%%%%%%%%%%%%%%%%%%%%%%%%%
%Cajas de colores
%
\input{/home/samuel/Documents/Latex/ColorBoxes}
\input{/home/samuel/Documents/Latex/Exercises}
%Ecuaciones resaltadas
% \usepackage[overload,ntheorem,reqno]{empheq}
%\input{/home/samuel/Documents/Latex/Ambientes-teoremas}
 \theoremstyle{plain}
 \renewcommand{\qedsymbol}{\makebox[7.7778pt][c]{\rule{1ex}{1ex}}}
 \newtheorem*{demo}{Demostración}
 \input{/home/samuel/Documents/Latex/TeoremasEnumerados}
\usepackage{hyperref}
\hypersetup{
    % bookmarks=false,         % show bookmarks bar?
    unicode=false,          % non-Latin characters in Acrobat?s bookmarks
    pdftoolbar=true,        % show Acrobat?s toolbar?
    pdfmenubar=true,        % show Acrobat?s menu?
    pdffitwindow=false,     % window fit to page when opened
    pdfstartview={FitH},    % fits the width of the page to the window
    pdftitle={Teoria de probabilidad},    % title
    pdfauthor={Samuel Ortega Cuadra},     % author
    pdfsubject={},   % subject of the document
    pdfcreator={Hecho con \LaTeX},   % creator of the document
    pdfproducer={ps2pdf}, % producer of the document
    pdfkeywords={} {} {}, % list of keywords
    pdfnewwindow=true,      % links in new window
    colorlinks=true,       % false: boxed links; true: colored links
    linkcolor=naranja,          % color of internal links
    citecolor=violeta,        % color of links to bibliography
    filecolor=magenta,      % color of file links
    urlcolor=cyan           % color of external links
}


\setcounter{secnumdepth}{4} %controla la profundidad de la numeración


\addto\captionsamerican{%
  \renewcommand{\contentsname}%
    {Índice}%
}

\title{Notas Probabilidad}
\author{Samuel Ortega Cuadra}
\input{/home/samuel/Documents/Latex/messages}

\begin{document}
    \begin{center}
        \huge{Teoría de Probabilidad. 1º de Carrera. Tema 2: Variables Aleatorias}
    \end{center}
    \tableofcontents
    \newpage
    \section{Introducción} % (fold)
    \label{sec:introducción}
        El objetivo de la propiedad, como hemos visto en el apartado anterior, busca estudiar los fenómenos aleatorios que resultan de un experimento. Sin embargo, muchas veces no nos interesan los restultados en si, sino una cantidad que nos permita extraer información de estos valores. Pongamos que estamos jugando a un juego de azar. No nos interesan tanto los resultados del juego como el beneficio o la pérdida que se pueda tener.
        \[\OMG = \{\text{Resultado del juego}\}\]
        Nos interesa: \[\OMG (resultado) \rightarrow \R(ganador) \]
        Por ejemplo, pongamos que jugamos con un dado trucado. Lo lanzamos muchas veces, pero el que anota el resultado solo comprueba si el resultado es par o impar. Los resultados posibles de este experimento serán $\OMG = \{1,2,3,4,5,6\}$. Digamos que el suceso $A = \text{"par"}$. 
        \begin{itemize}
            \item Obtenemos como resultado: \[\prob(A) \approx \frac{7}{10}\]
            \item Información: $\sigma-algebra$ \[\mathcal{A} = \{\emptyset, \OMG, A, A^c\}\]
            \item Observación: no podemos calcular $\prob(1)$
            \item P queda determinada por $\prob(A) \approx \frac{7}{10}$
            \item Apuesta: Se gana 10 euros si sale un \textbf{5 o 6}. Se pierden 2 euros si sale cualquier otro número.
            \item Aplicación:\[X : \OMG \rightarrow \R\]
            \[X(5) = X(6) = 10\text{ \euro }; X(1) = X(2) = X(3) = X(4) = -2 \text{ \euro }\]
        \end{itemize}
        En nuestro modelo $(\OMG,\mathcal(A),\prob)$, \textbf{NO} podemos calcular la probabilidad de ganar 10\euro:
        \[\prob(\text{ganar 10 \euro}) = \prob(\{w\in\OMG,X(\omega) = 10\}) = \prob(\{5,6\})\ \rightarrow \text{No pertenece a $\mathcal{A}$}\]
        \textbf{DEFINICIÓN:} dado $(\OMG,\mathcal(A),\prob)$, diremos que una aplicación \[X : \OMG \rightarrow \R\] es una \textbf{variable aleatoria} si, para todo intervalo $B\in\R$ (o semirecta) tenemos que:
        \[\{\omega\in\OMG, X(\omega) \in B\} \in \mathcal{A}\]
        En el caso mencionado anteriormente, $X$ no es una variable aleatoria, ya que el resultado sería $\{5,6\}$, que no pertenece a $\mathcal{A}$\\
        Observaciones:
        \begin{enumerate}
            \item En la definición de variable aleatoria no interviene $\prob$, pese a que la motivación es calcular:
            \[\prob(\underbrace{\{\omega\in\OMG, X(\omega) \in B\}}_{X^{-1}(B)}) = \prob(X\in B)\]
            \item Notación: Si $B = \{a\}$, entonces:\[\prob(X\in B) = \prob(X\in\{a\}) = \prob(X=a)\] o también:
            \[a \le b: B = [a,b]: \prob(X\in [a,b]) = \prob(a\le X\le b)\]
            \item Si $\mathcal{A} = P(\OMG)$, entonces toda aplicación $X : \OMG \rightarrow \R$ es variable aleatoria.
        \end{enumerate}
    %sec: introduccion(end)
    \section{Como hallar variables aleatorias} % (fold)
    \label{sec:como_hallar_variables_aleatorias}
        \subsection{Operaciones} % (fold)
        \label{sub:operaciones}
            Sean $X,Y$ variables aleatorias y $a\in\R$ entonces son variables aleatorias:
            \begin{itemize}
                \item $X + Y$
                \item $X \cdot Y$
                \item $a\cdot X$
                \item $abs{X}$
                \item $X^a$
                \item Si $X(\omega)\ne 0; \forall \omega \in \OMG$, entonces $\dfrac{1}{X}$ también es una variable aleatoria.
            \end{itemize}
        % subsection operaciones (end)
        \subsection{Sucesiones} % (fold)
        \label{sub:sucesiones}
            Tomemos $\{X_n, n\ge1\}$ como una sucesión de variables aleatorias tales que $\forall \omega \in \OMG$, la sucesión
            \[\{X_n(\omg)\in \R, n\ge1\}\]
            es convergente. En ese caso, definiremos\[X(\omega) \coloneqq \lim_{n\to\infty}X_n(\omega) \in \R\]
            Entonces, $X$ es una variable aleatoria.
        % subsection sucesiones (end)    
    % section como_hallar_variables_aleatorias(end)
    \section{Función de distribución de una variable aleatoria} % (fold)
    \label{sec:función_de_distribución_de_una_variable_aleatoria}
    \textbf{DEFINICIÓN:} Sea $X$ una variable aleatoria. La \textbf{función de distribución} de $X$ es:
    \[F:\R \rightarrow [0,1]\]
    \[x \rightarrow F(x) = \prob(X\le x) = \prob(\{\omg \in \OMG, X(\omg) \le x\} = \prob(\underbrace{\{\omg \in \OMG, X(\omg) \in B\}}_{\in \mathcal{A}}) \text{ donde }B = (-\infty,x]\]
    \subsection{Propiedades} % (fold)
    \label{sub:propiedades}
        \begin{enumerate}
            \item F es no decreciente: $x<y \Rightarrow F(x) \le F(y)$
            \item F es continua por la derecha y tiene límite por la izquierda en todos sus puntos (càdlàg \textit{continu à droit, limit à gauche}).
            \item \[\lim_{x\to-\infty}F(x) = 0\;;\;\lim_{x\to+\infty}F(x) = 1\]
            \item F tiene como máximo un numero numerable de puntos de discontinuidad. Esto se debe a que la recta real se puede expresar como la unión de $\mathbb{Z}$ intervalos.
            \item \[\prob(s\le X \le t) = F(t) - F(s)\]
            \item \[\prob(X < x) = F(x^-) = \lim_{X\to x^-}F(X)\]
            \item \[\prob(X=x) = F(x) - F(x^-)\]
            si $F$ es discontinua en $x$:
                \[\prob(X=x) > 0\]
        \end{enumerate}
    % subsection propiedades (end)
    % section función_de_distribución_de_una_variable_aleatoria (end)
    \section{Variables aleatorias discretas} % (fold)
    \label{sec:variables_aleatorias_discretas}
        Una variable aleatoria $X$ es \textbf{discreta} si:
        \[\exists S \subset R \text{ finito o numerable tal que } \prob(X\in S) = 1\]
        Siendo $S$ el soporte de $X$. Suponemos que todo $x\in S$ cumple que $\prob(X=a) > 0$. También:
        \[S = \{x_i, i\in I\} ; I\subseteq \N = \{1,2,3,...\}\]
        \subsection{Función de probabilidad} % (fold)
        \label{sub:función_de_probabilidad}
            \[p:S \rightarrow [0,1]\]
            \[x_i \rightarrow p(x_i) = \prob(X=x_i)\]
        En la notación de esta función normalmente se escribirá $p_i$ para referirse a $p(x_i)$.\\
        Observaciones:
        \begin{enumerate}[label=\Alph*]
            \item \[\sum_{i\in I} p_i = \sum_{i\in I} \prob(X = x_i) = \prob(X \in S) = 1\]
            \item $\forall B \subset \R$ intervalo/semirecta 
            \[\prob(X\in B) = \sum_{i\in I, x\in B} p_i\]
            \item La función de probabilidad de X permite calcular todas las probabilidades relacionadas con la variable aleatoria X y determina la \textbf{función de distribución}:
            \[F(x) = \prob(X\le x) = \sum_{i\in I, x_i \le x} pi\]
        \end{enumerate}
        % subsection función_de_probabilidad (end)
        \subsection{Tipos de variable discreta} % (fold)
        \label{sub:tipos_de_variable_discreta}
            \subsubsection{Caso Degenerado} % (fold)
            \label{subsub:caso_degenerado}
                \[X = a \Rightarrow S = \{a\} \text{ y } \prob(X=a) = 1\]
            % subsubsection caso_degenerado (end)
            \subsubsection{Caso Bermoulli} % (fold)
            \label{subsub:caso_bermoulli}
                \[X = \{1, \text{ con probabilidad }p\in(0,1)\text{ y } 0,\text{ con probabilidad } 1 -p\in(0,1) \}\]
                \[S = \{a\}; \prob(X=1) = p, \prob(X=0) = 1 - p \]
                X tiene distribución Bermouilli de parámetro $p$:
                \[X \sim B(p) \]
            % subsubsection caso_bermoulli (end)
            \subsubsection{Uniforme discreta sobre n puntos} % (fold)
            \label{subsub:uniforme_discreta_sobre_n_puntos}
                Toman valores en $\{x_1,x_2,x_3,...,x_n\}$ con la misma probabilidad. Tenemos que:
                \[\prob(X=x_i) = \frac{1}{n}\]
                Denotamos:
                \[X \sim U\{x_1,...,x_n\}; \; S=\{x_1,...,x_n\} ; \; p(x_i) = \frac{1}{n}, \forall i=1,...,n\]

            % subsubsection uniforme_discreta_sobre_n_puntos (end)
            \subsubsection{Binomial} % (fold)
            \label{subsub:binomial}
            \begin{itemize}
                \item Repetimos n veces el mismo experimento
                \item Tenemos un suceso $A$ con $p=\prob(A)$. Si $A$ se realiza es un éxito, sino es un fracaso.
                \item \[X = \text{Numero de veces que ocurre $A$ en las $n$ repeticiones}; \; S = \{0,1,2,3,4,...,n\}\]
                \[\prob(X=k) = \binom{n}{k} p^k \cdot (1-p)^{n-k}, k=0,1,...,n\]
                \item Denotamos como:
                \[X\sim B(n,p)\]
            \end{itemize}
            \underline{Observación:} Pongamos que tenemos una serie de variables aleatorias $X_1,X_2,...,X_n$ donde
            \begin{equation}
                X_i = 
                \begin{cases}
                    1, \text{Si en la i-ésima repetición se realiza $A$}\\
                    0, \text{En caso contrario}
                \end{cases}
            \end{equation}
            Entonces:
            \[X_i = B(p)\]
            Tenemos que:
            \[X = \sum_{i=1}^n X_i \sim B(n,p) \text{ (repeticiones independientes)}\]
            % subsubsection binomial (end)
            \subsubsection{Geométrica} % (fold)
            \label{subsub:geometrica}
                Pongamos que tenemos un experimento aleatorio con una probabilidad:
                \[p = \prob(\text{éxito})\]
                \[X = \text{"número de pruebas hasta obtener el primer éxito (incluida la prueba con el éxito)"}\]
                \[S = \{1,2,3,...\}\]
                \textbf{Función de probabilidad:}
                \[P(X=k) = (1-p)^{k-1} \cdot p\]
                \textbf{Notación:} \[X_n \sim Geom(p)\]
                \textbf{Propiedades:}
                \[\prob(X>l+k | X>l) = \prob(X>k), \; \forall k,l\in\N \]
                Esta propiedad es conocida como falta de memoria. Para entenderlo puede pensarse de la siguiente forma. Supongamos que estás lanzando una moneda desde las 9 de la mañana hasta las 11, y no sale ninguna cara. Si alguien llegara a ĺas 11 y los dos empezarais a lanzar monedas en ese instante, quien de los dos tendría ¿más posibilidades de sacar una cara antes? Los dos tendrían la misma.
            %subsubsection geometrica (end)
            \subsubsection{Binomial negativa} % (fold)
            \label{subsub:binomial_negativa}
                \[X = \text{número de pruebas hasta que obtenga $m$ éxitos}\]
                \textbf{Observación:} Si $m=1 \Rightarrow X_n \; Geom(p)$\\
                En general:
                \[S =  \{m,m+1,m+2,...\} \text{ con } \prob(X=k) = \binom{k-1}{m-1}(1-p)^{k-m}\cdot p^m \]
                \textbf{Notación:}\[X_n \sim BN(m,p)\]
            %subsubsection binomial_negativa (end)
            \subsubsection{Hipergeométrica} % (fold)
            \label{subsub:hipergeometrica}
            Pongamos que tenemos una urna con $N$ bolas. $B$ bolas blancas y $N-B$ rojas. De esta urna sacamos $n\le N$ bolas. 
            \[X = \text{"número de bolas blancas obtenidas"}\]
            Si estamos trabajando con reemplazamiento:
            \[X \sim B(n, \frac{B}{N})\]
            Si trabajamos sin reemplazamiento:
            \[\prob(X=j) = \frac{\frac{n!}{(n-j)!}}{\binom{N}{n}}\]
            Es más, podemos deducir que:
            \[max(n-N+B,0) \le j \le min(B,n)\]
            \textbf{Notación:}\[H\sim H(N,B,n)\]

            Pongamos un ejemplo. En una mesa de 10 personas hay 4 menores de edad. Todas las bebidas que se beben son alcoholicas y el camarero pide 5 DNI al azar.
            \begin{itemize}
                \item ¿$\prob($Descubrir los 4 menores de edad$)$?:
                \[N = 10 \; ; \; B = 4 \; ; \; N-B=6 \; ; \; n = 5\]
                \[X = \text{"Número de menores de edad"} \Rightarrow X \sim H(10,4,5)\]
                \[\prob(X=4) = \frac{\frac{5!}{(5-4)!}}{\binom{10}{4}} = 0.024\]
                \item ¿Cuantos DNI se deben pedir para que la probabilidad de detectar a los 4 menores sea minimo 0.25?
                \[n=7 \Rightarrow \prob(X=4) = \frac{\frac{7!}{(7-4)!}}{\binom{10}{7}} = 0.16667\]
                \[n=8 \Rightarrow \prob(X=4) = \frac{\frac{8!}{(8-4)!}}{\binom{10}{8}} = 0.3333 > 0.25\]
            \end{itemize}
            %subsubsection hipergeometrica (end)
            \subsubsection{Poisson}
            \label{subsub:poisson}
                $\lambda \in \R$. Se dice que $X$ tiene ley de \textbf{Poisson de parámetros $\lambda$} si se cumple que $\N \cup \{u\}$ y
                \[\prob(X=k) = e^{-\lambda} \cdot \frac{\lambda^{k}}{k!}, \; k=0,1,2...\]
                Se escribe:
                \[X \sim Poiss(\lambda)\]
                Esta variable aleatoria aparece cuando se busca calcular el número de veces que ocurre un suceso aleatorio en un parámetro de tiempo fijo. Más adelante veremos que:
                \[\lambda = \text{Valor medio de los sucesos que contamos } (por \;unidad \;de \; tiempo/espacio) \]


                \textbf{Aproximación de la Binomial por la Poisson:} Pongamos que tenemos una sucesión:
                \[\{p_n,n\ge 1\} \text{ tal que } 0 < p_n < 1 \; y \; \lim_{n\to\infty} np_n = \lambda > 0 \Rightarrow \lim_{n\to\infty} p_n = 0\]
                También se deduce que:
                \[\lambda > 0 \Rightarrow p_n = \frac{\lambda}{n}\]
                \[X_n \sim B(n,p_n)\]
                \[\prob(X_n = k) = \binom{n}{k} p_n^k (1-p_n)^{n-k}, \; k=0,1,...,n\]
                Si $k=0$:
                \[\lim_{n\to\infty} \prob(X_n = 0) = \lim_{n\to\infty} (1-p)^n = \lim_{n\to\infty} (1+\frac{1}{-\frac{1}{p_n}})^{n(-p_n)\cdot(\frac{1}{-p_n})}\]
                \[=\lim_{n\to\infty} [(1 + \frac{1}{-\frac{1}{p_n}})^{-\frac{1}{p_n}}]^{-np_n} = e^\lambda\]
                Si $k\ge1$ para $n$ suficientemente grande:
                \[\prob(X_n=k) = \frac{n!}{(n-k)!k!} p_n^k (1-p_n)^{n-k} = \frac{n\cdot(n-1) \cdots (n-k+1)}{n^k} \frac{(np_n)^k}{k!} (1-p_n)^n (1-p_n)^{-k} \]
                \[\xRightarrow{n \to \infty} prob(X=k) = 1 \cdot \frac{\lambda^{k}}{k!} \cdot e^{-\lambda} \cdot 1 = \rightarrow X \sim Poiss(\lambda)\]


                 \textbf{Ejemplo:} Tenemos un lote de 200 unidades.
                 \[X = \text{"Unidades defectuosas"}\]
                 \[X \sim B(200,p)\; \{p_1 = 0.001 ; \; p_2 = 0.05\} \]
                 \[\prob(\text{Aceptar el lote}) = \prob(X \le 2)\]
                 Si $p=0.001$:
                 \[\prob(\text{Aceptar el lote}) = \prob(X \le 2) \sim \prob(Y\le 2)\]
                 Siendo $Y \sim Poiss(0.001\times200)$:
                 \[=\prob(Y=0)+\prob(Y=1)+\prob(Y=2) = \]
            %subsubsection poisson (end)
        % subsection tipos_de_variable_discreta (end)
    % section variables_aleatorias_discretas (end)
    \section{Variables aleatorias continuas} % (fold)
    \label{sec:variables_aleatorias_continuas}
        Una variable aleatoria $X$ es \textbf{continua} si $\exists f:\R\rightarrow\R$:
        \begin{itemize}
            \item La función es no negativa.
            \[f(x) \ge 0, \forall x \in \R\]
            \item f es integrable en $\R$ y \[\int_{-\infty}^{+\infty} f(x) dx = 1\]
            \item La probabilidad de que la variable tome valores entre a y b es la integral entre ambos valores.\[\forall -\infty \le a \le b \le +\infty\; \prob(a\le X\le b) = \int_{a}^b f(x) dx\]
            \item El valor $f$ se denomina \textbf{densidad de $X$}.
            \item La probabilidad de que la variable tome un valor exacto es $0$
            \[\prob(X=a) = \prob(a \le X \le a) = \int_{a}^a f(x) dx= 0, \; \forall a\in\R\]
            \subsection{Función de distribución} % (fold)
            \label{sub:función_de_distribución}
                Para calcular la función de distribución de una variable aleatoria continua tomamos la siguiente función:
                \[F(x) = \prob(X\le x)= \prob(-\infty < X \le x) = \int_{-\infty}^x f(t) dt\]
                Esto implica que no solo $F(x)$ es continua sino que además es absolutamente continua.
        \end{itemize}
            % subsection función_de_distribución (end)
            \subsection{Tipos de variable continua} % (fold)
            \label{sub:tipos_de_variable_continua}   
                \subsubsection{Uniforme en (0,1)} % (fold)
                \label{subsub:uniforme_en_}
                    \textbf{NOTACIÓN:} \[X \sim U(0,1)\]
                    Esta variable aleatoria representa el resultado de escoger un número al azar del intervalo $(0,1)$ (uniformemente)\\
                    \textbf{Densidad:}
                    \begin{equation}
                        f(x) =
                        \begin{cases}
                            1, \; x\in(0,1)\\
                            0, \; \text{en caso contrario}
                        \end{cases} \nonumber
                    \end{equation}
                    Para calcular la probabilidad entre un intervalo dado $a,b$ empleariamos la integral:
                    \[\prob(a\le X\le b) = \int_a^b f(x) dx = \int_a^b 1 dx = b-a\]

                    \textbf{Función de distribución:}
                    \begin{equation}
                        F(x) = \prob(X\le x) = \int_{-\infty}^x f(t) dt =
                        \begin{cases}
                            0, \; si \; x<0\\
                            \int_0^x 1 dt = x, \; si \; 0\le x \le 1\\
                            1, \; si \; x\ge 1\\
                        \end{cases}
                        =
                        \begin{cases}
                            0, \; si \; x<0\\
                            x, \; si \; 0\le x<1\\
                            1, \; si \; x \ge 1
                        \end{cases}\nonumber
                    \end{equation}
\end{document}