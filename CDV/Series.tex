\documentclass[11pt]{article}
\RequirePackage{etex}
%\pagestyle{empty}
\usepackage[activeacute,spanish,american]{babel}
\usepackage[utf8]{inputenc}%Para usar los acentos normalmente.
\usepackage[T1]{fontenc}
% \usepackage[usenames,dvipsnames,svgnames]{xcolor}
\usepackage{fullpage}
\usepackage{graphicx}
\usepackage{url}
\urlstyle{same}

\usepackage[makestderr]{pythontex}
% \restartpythontexsession{\thesection}

%%%%%%%%%%%%%%%%%%%%%%%%%%%%%%%%%%%%%%%%%%%%%%%%%%%%%%%5

\usepackage{tkz-base}
\usepackage[framemethod=TikZ]{mdframed}
\usepackage[most]{tcolorbox}
\usepackage{helvet, amssymb,amsmath,latexsym}
\usepackage{amsthm}
\usepackage{mathtools}
\usepackage{textcomp}
\usepackage{multienum}
\usepackage[inline,shortlabels]{enumitem}
\usepackage{multicol}
\usepackage{float}
% \usepackage{gensymb}
\providecommand{\norm}[1]{\left\lVert #1 \right \rVert}
\providecommand{\abs}[1]{\left\lvert #1\right\rvert}
\usepackage{color,soul}%permite texto y subrayar en color.
\input{/home/samuel/Documents/Latex/Colores.tex}
% \usepackage[pdftex]{graphicx}
%Dimensiones
\usepackage[a4paper,left=2cm,right=1.5cm, top=1.5cm, bottom=3cm,includehead]{geometry}
%%%%%%%%%%%%%%%%%%%%%%%%%%%%%%%%%%%%%%%%%%%%%%%%%%%%%%%%%%%%%%%%%%%%%%%%%%%%%%%%%%%%%%%%%%%%
%Nombres de conjuntos y comandos propios.
\providecommand{\norm}[1]{\left\lVert #1 \right \rVert}
\providecommand{\abs}[1]{\left\lvert #1\right\rvert}
\newcommand{\C}{\mathbb{C}}
\newcommand{\N}{\mathbb{N}}
\newcommand{\Q}{\mathbb{Q}}
\newcommand{\Z}{\mathbb{Z}}
\newcommand{\R}{\mathbb{R}}
\newcommand{\I}{\mathbb{I}}
\newcommand{\prob}{\mathbb{P}}
\newcommand{\omg}{\omega}
\newcommand{\OMG}{\varOmega}
% \usepackage{esvect}
% \renewcommand{\vec}{\vv}
 \usepackage{pgf,tikz}
\usetikzlibrary{arrows.meta,arrows}
\usetikzlibrary{shadows}
\usetikzlibrary{shapes}
\usetikzlibrary{decorations.pathmorphing}
\usetikzlibrary{shapes.multipart}
\usetikzlibrary{chains}
\usetikzlibrary{scopes}
\usetikzlibrary{matrix}
\usetikzlibrary{positioning,automata,calc}
 %\usepackage{framed}{}
 %\usepackage[framed, amsthm,thmmarks,thref]{ntheorem}
 %\usepackage{tkz-tab,tkz-euclide,tkz-fct,tkz-linknodes}
 \usepackage{tkz-tab,tkz-euclide}
 \usepackage{pythontex}
% \usetkzobj{all}
%%%%%%%%%%%%%%%%%%%%%%%%%%%%%%%%%%%%%%%%%%%%%%%%%%%%%%%%%%%%%%%%%%
%Cajas de colores
%
\input{/home/samuel/Documents/Latex/ColorBoxes}
\input{/home/samuel/Documents/Latex/Exercises}
%Ecuaciones resaltadas
% \usepackage[overload,ntheorem,reqno]{empheq}
%\input{/home/samuel/Documents/Latex/Ambientes-teoremas}
 \theoremstyle{plain}
 \renewcommand{\qedsymbol}{\makebox[7.7778pt][c]{\rule{1ex}{1ex}}}
 \newtheorem*{demo}{Demostración}
 \input{/home/samuel/Documents/Latex/TeoremasEnumerados}
\usepackage{hyperref}
\hypersetup{
    % bookmarks=false,         % show bookmarks bar?
    unicode=false,          % non-Latin characters in Acrobat?s bookmarks
    pdftoolbar=true,        % show Acrobat?s toolbar?
    pdfmenubar=true,        % show Acrobat?s menu?
    pdffitwindow=false,     % window fit to page when opened
    pdfstartview={FitH},    % fits the width of the page to the window
    pdftitle={CheatSheet Numérico},    % title
    pdfauthor={Suiza},     % author
    pdfsubject={},   % subject of the document
    pdfcreator={Hecho con \LaTeX},   % creator of the document
    pdfproducer={ps2pdf}, % producer of the document
    pdfkeywords={} {} {}, % list of keywords
    pdfnewwindow=true,      % links in new window
    colorlinks=true,       % false: boxed links; true: colored links
    linkcolor=naranja,          % color of internal links
    citecolor=violeta,        % color of links to bibliography
    filecolor=magenta,      % color of file links
    urlcolor=cyan           % color of external links
}


\setcounter{secnumdepth}{4} %controla la profundidad de la numeración

\addto\captionsamerican{%
  \renewcommand{\contentsname}%
    {Índice}%
}


\title{CheatSheet Series}
\author{Suiza}
\input{/home/samuel/Documents/Latex/messages}

\begin{document}
    \begin{center}
        \huge{CheatSheet Series}
    \end{center}
    \section{Fórmulas Trigonometría Hiperbólicas} % (fold)
    \label{sec:formulas_trigonometria_hiperbolico}
    Coseno hiperbólico:
    \[cosh(x) = \sum_{k=0}^{\infty} \frac{x^{(2k)}}{(2k)!}\]
    Seno hiperbólico:

    \[sinh(x) = \sum_{k=0}^{\infty} \frac{x^{(2k +1)}}{(2k+1)!}\]
    % section formulas_trigo_hiperbólico (end)
    \section{Series geométricas} % (fold)
    \label{sec:series_geométricas}
      \[\text{Serie geométrica de razón p} \rightarrow \sum_{k=0}^{\infty} p^k\]
      Si el valor absoluto de la razón es menor que 1:
      \[\sum_{k=0}^{\infty} p^k = \frac{1}{1-p}\]
      Suma de los $N$ primeros elementos de una serie geométrica empezando desde a:
      \[\sum_{k=a}^{N} p^k = \frac{p^a - p^{N+1}}{1 - p}\]
    % section series_geométricas (end)
    \section{Exponencial} % (fold)
    \label{sec:exponencial}
      \[e^x = \sum_{k=0}^{\infty} \frac{x^k}{k!}\]
    \subsection{Arreglos de la exponencial} % (fold)
    \label{sub:arreglos_de_la_exponencial}
      \[\sum_{k=1}^{\infty} \frac{x^k}{k!} =  -1 + \sum_{k=0}^{\infty} \frac{x^k}{k!} \]
    % subsection arreglos_de_la_exponencial (end)
    % section exponencial (end)
    \section{Probabilidad} % (fold)
    \label{sec:probabilidad}
    \subsection{Poisson} % (fold)
    \label{sub:poisson}
    Si tenemos una distribuciónd de Poisson de parámetro $\lambda$
      \[\prob(X = k) = e^{-\lambda} \cdot \frac{\lambda^k}{k!}\]
      Si se quiere aproximar una binomial a una poisson:
      \[B(n,p_n) \sim Poiss(n\cdot p_n)\]
    % subsection poisson (end)
    \subsection{Unión de 3 elementos} % (fold)
    \label{sub:unión_de_3_elementos}
    \[\prob(X_1 \cup X_2 \cup X_3) = \prob(X_1) + \prob(X_2) + \prob(X_3) - \prob(X_1 \cap X_2) - \prob(X_1 \cap X_3) - \prob(X_2 \cap X_3) + \prob(X_1 \cap X_2 \cap X_3) \]
    El caso para n:
    \[\prob(X_1 \cup X_2 \cup X_3 \cup...) = \text{(Prob. Individuales)} - \text{(Prob. $\cap$ parejas)} + \text{(Prob. $\cap$ trios)} - ... \]
    % subsection unión_de_3_elementos (end)
    % section probabilidad(end)

\end{document}